% Created 2023-04-24 Mon 17:10
% Intended LaTeX compiler: pdflatex
\documentclass[presentation]{beamer}
\usepackage[utf8]{inputenc}
\usepackage[T1]{fontenc}
\usepackage{graphicx}
\usepackage{longtable}
\usepackage{wrapfig}
\usepackage{rotating}
\usepackage[normalem]{ulem}
\usepackage{amsmath}
\usepackage{amssymb}
\usepackage{capt-of}
\usepackage{hyperref}
\usetheme{default}
\author{Mathematical Logic Course}
\date{\today}
\title{Week 10: Set Theory and the Continuum Hypothesis}
\hypersetup{
 pdfauthor={Mathematical Logic Course},
 pdftitle={Week 10: Set Theory and the Continuum Hypothesis},
 pdfkeywords={},
 pdfsubject={},
 pdfcreator={Emacs 28.2 (Org mode 9.5.5)}, 
 pdflang={English}}
\begin{document}

\maketitle

\begin{frame}[label={sec:orgfb3648e}]{Introduction}
\begin{itemize}
\item Welcome to Week 10 of our Mathematical Logic Course!
\item This week, we'll explore Set Theory and the Continuum Hypothesis.
\item We'll cover the following topics:
\begin{itemize}
\item Introduction to set theory: Zermelo-Fraenkel axioms
\item The Continuum Hypothesis: statement and independence
\end{itemize}
\end{itemize}
\end{frame}

\begin{frame}[label={sec:org19d0cd0}]{Set Theory: Zermelo-Fraenkel Axioms}
\begin{itemize}
\item What is set theory?
\item Zermelo-Fraenkel axioms: foundation of modern set theory
\item Understanding axiomatic set theory and its role in mathematics
\item Examples of sets, operations, and relations
\end{itemize}
\end{frame}

\begin{frame}[label={sec:orgd1dadfa}]{Infinite Sets: Countable and Uncountable}
\begin{itemize}
\item Introduction to infinite sets
\item Distinction between countable and uncountable sets
\item Examples of countable and uncountable sets
\item Cardinality of infinite sets: aleph numbers
\end{itemize}
\end{frame}

\begin{frame}[label={sec:orgbaf64c7}]{The Continuum Hypothesis}
\begin{itemize}
\item Statement of the Continuum Hypothesis (CH)
\item Understanding the significance of the CH in set theory
\item Exploring the concept of the cardinality of the continuum
\end{itemize}
\end{frame}

\begin{frame}[label={sec:org5b764ea}]{Independence of the Continuum Hypothesis}
\begin{itemize}
\item Gödel and Cohen's work on the independence of the CH
\item Understanding the concept of independence in mathematical logic
\item Implications for set theory and mathematical reasoning
\end{itemize}
\end{frame}

\begin{frame}[label={sec:org7662f44}]{Summary and Conclusion}
\begin{itemize}
\item Recap of the topics covered in this lecture
\item Set theory and its foundational role in mathematics
\item The Continuum Hypothesis and its independence
\item Next week, we'll explore Model Theory and Nonstandard Models
\end{itemize}
\end{frame}

\begin{frame}[label={sec:org664e513}]{Questions and Discussion}
\begin{itemize}
\item Do you have any questions about today's lecture?
\item Let's discuss the material and explore any questions you may have
\end{itemize}
\end{frame}

\begin{frame}[label={sec:orga6a2f46}]{Coding Exercises}
\begin{itemize}
\item Implementing set operations and exploring infinite sets in Python
\item Understanding the Continuum Hypothesis through coding exercises
\end{itemize}
\end{frame}
\end{document}