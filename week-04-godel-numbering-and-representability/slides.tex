% Created 2023-04-24 Mon 17:10
% Intended LaTeX compiler: pdflatex
\documentclass[presentation]{beamer}
\usepackage[utf8]{inputenc}
\usepackage[T1]{fontenc}
\usepackage{graphicx}
\usepackage{longtable}
\usepackage{wrapfig}
\usepackage{rotating}
\usepackage[normalem]{ulem}
\usepackage{amsmath}
\usepackage{amssymb}
\usepackage{capt-of}
\usepackage{hyperref}
\usetheme{default}
\author{Mathematical Logic Course}
\date{\today}
\title{Week 4: Gödel Numbering and Representability}
\hypersetup{
 pdfauthor={Mathematical Logic Course},
 pdftitle={Week 4: Gödel Numbering and Representability},
 pdfkeywords={},
 pdfsubject={},
 pdfcreator={Emacs 28.2 (Org mode 9.5.5)}, 
 pdflang={English}}
\begin{document}

\maketitle

\begin{frame}[label={sec:orgcaf1d4f}]{Introduction}
\begin{itemize}
\item Welcome to Week 4 of our Mathematical Logic Course!
\item This week, we'll explore Gödel numbering and the concept of representability.
\item We'll cover the following topics:
\begin{itemize}
\item Introduction to Gödel numbering
\item Representability of recursive functions in formal systems
\item Examples and applications
\end{itemize}
\end{itemize}
\end{frame}

\begin{frame}[label={sec:orge4d01c5}]{Gödel Numbering}
\begin{itemize}
\item What is Gödel numbering?
\item Kurt Gödel's contributions to mathematical logic
\item Encoding formulas, proofs, and sequences as natural numbers
\item The significance of Gödel numbering in the Incompleteness Theorems
\end{itemize}
\end{frame}

\begin{frame}[label={sec:org9ef848a}]{Representability}
\begin{itemize}
\item What does it mean for a function to be representable in a formal system?
\item Recursive functions and their representability
\item Formalizing mathematical theories (e.g., Peano Arithmetic)
\item The concept of a complete and consistent formal system
\end{itemize}
\end{frame}

\begin{frame}[label={sec:orgba64cd7}]{Examples of Representability}
\begin{itemize}
\item Example: Representing addition and multiplication in a formal system
\item Example: Representing the successor function in Peano Arithmetic
\item Example: Representing the prime function in a formal system
\end{itemize}
\end{frame}

\begin{frame}[label={sec:org36f0141}]{Summary and Conclusion}
\begin{itemize}
\item Recap of the topics covered in this lecture
\item Gödel numbering and the concept of representability
\item The foundational role of Gödel numbering in Gödel's Incompleteness Theorems
\item Next week, we'll dive into Gödel's First Incompleteness Theorem
\end{itemize}
\end{frame}

\begin{frame}[label={sec:org12578bb}]{Questions and Discussion}
\begin{itemize}
\item Do you have any questions about today's lecture?
\item Let's discuss the material and explore any questions you may have
\end{itemize}
\end{frame}

\begin{frame}[label={sec:orgbd2d1bb}]{Coding Exercises}
\begin{itemize}
\item Generating Gödel numbers for formulas and exploring representability in Python
\item Implementing recursive functions and testing their representability
\end{itemize}
\end{frame}
\end{document}